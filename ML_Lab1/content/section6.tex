\pagebreak
\section{Mô tả ứng dụng}
\subsection{Phân tích và trực quan hoá mức độ ảnh hưởng của thuộc tính}
\subsubsection{Mục tiêu}
Mục tiêu chính của hệ thống là cho phép người dùng trải nghiệm trực tiếp tính năng dự đoán của mô hình, đồng thời giúp họ hiểu được phần nào cách thức mà mô hình đưa ra dự đoán. Chính vì thế, giao diện ứng dụng sẽ đưa ra các cơ chế trực quan hoá để làm rõ mối quan hệ giữa các biến đầu vào và kết quả dự đoán giá xe hay phân tích sự ảnh hưởng của các biến lên giá xe sau cùng.
\subsubsection{Triển khai}
Khi người dùng chọn một mô hình trên giao diện, ứng dụng sẽ tải tệp mô hình \texttt{.pkl} tương ứng. Quá trình trích xuất thông tin diễn ra như sau:
\begin{itemize}
    \item \textbf{Trích xuất hệ số:} Các mô hình hồi quy tuyến tính lưu trữ tầm quan trọng của mỗi thuộc tính dươi dạng mảng các hệ số (\texttt{coef\_}). Trích xuất mảng này từ pipeline của mỗi mô hình.
    \item \textbf{Trích xuất tên thuộc tính:} Ánh xạ các hệ số vừa trích xuất xong về lại tên ban đầu của nó bằng cách truy cập vào bước tiền xử lý trong pipeline:
    \begin{itemize}
        \item Đối với các mô hình chính quy hoá (Ridge, Lasso, Elastic Net), tên thuộc tính được trích xuất từ thuộc tính \texttt{feature\_columns\_} của bước \texttt{"preprocess"} (lớp \texttt{TabularPreprocessor}).
        \item Đối với mô hình hồi quy tuyến tính sử dụng BIC, tên đặc trưng được trích xuất từ phương thức \texttt{get\_feature\_names\_out()} của bước chọn lọc thuộc tinh, đảm bảo chỉ có các thuộc tính được lựa chọn mới hiển thị.
    \end{itemize}
    \item \textbf{Trực quan hoá:} Sau khi có được danh sách các tên thuộc tính ánh xạ với mảng chỉ số \texttt{coef\_}, kết hợp chúng thành một cấu trúc dữ liệu \texttt{DataFrame} duy nhất, sau đó dùng \texttt{Altair} để vẽ biểu đồ. Biểu đồ được sắp xếp dựa trên giá trị tuyệt đối của hệ số và chỉ hiển thị 20 đặc trưng có ảnh hưởng lớn nhất.
\end{itemize}
\begin{figure}[H]
    \centering
    \includegraphics[width=0.6\textwidth]{img/demo_influence.png}
    \caption{Biểu đồ ví dụ thể hiện độ ảnh hưởng của thuộc tính lên kết quả dự đoán sau cùng được trích xuất từ giao diện ứng dụng web.}
    \label{fig:mylabel}
\end{figure}
Trong biểu đồ trên, các thuộc tính có đường biểu diễn chỉ số tầm ảnh hưởng màu xanh tức là chúng có ảnh hưởng dương hay đồng biến với giá xe (giá xe sẽ tăng nếu các thuộc tính này tăng) và các thuộc tính có chỉ số tầm ảnh hưởng màu đỏ nghĩa là chúng có ảnh hưởng âm hay nghịch biến đến giá xe sau cùng.\\
Ngoài ra, giao diện ứng dụng web cũng cung cấp một cách xem khác để có thể thấy được chi tiết toàn bộ thuộc tính và chỉ số tầm ảnh hưởng của chúng, được cài đặt để hiển thị khi người dùng ấn nút "See all feature coefficients".
\begin{figure}[H]
    \centering
    \includegraphics[width=0.6\textwidth]{img/demo_influence_table.png}
    \caption{Bảng chi tiết ví dụ thể hiện độ ảnh hưởng của thuộc tính lên kết quả dự đoán sau cùng được trích xuất từ giao diện ứng dụng web.}
    \label{fig:mylabel}
\end{figure}
Như hình trên, giao diện bảng này cung cấp một nút tải xuống để cho người dùng có thể xuất bảng ra file \texttt{.csv} để dễ dàng thực hiện các nghiệp vụ tính toán khác.

\subsection{Tổng quan}
Nhóm đã xây dụng một ứng dụng web đơn giản dùng để tương tác với mô hình, cho phép nhập dữ liệu riêng về một chiếc xe và chọn model dự đoán giá của chiếc xe đó. Ngoài ra tùy vào lựa chọn mô hình mà người dùng cũng có thể xem các biểu đồ về sức ảnh hưởng của các attribute lên dự đoán của chúng.
\subsection{Các công nghệ đã sử dụng}
Ứng dụng được xây dựng hoàn toàn bằng Python với các thư viện chính:
\begin{itemize}
    \item \textbf{Streamlit: }Đây là framework nhẹ dùng để xây dựng một giao diện web tương tác với người dùng, thích hợp để trình bày các dữ liệu, mô hình, kết quả trong lĩnh vực AI/ML và Data Science.
    \item \textbf{Joblib: }Dùng để đóng gói quá trình preprocess và bản thân mô hình đã được huấn luyện thành các file .pkl, đồng thời dùng để tải các file đóng gói đó lên web để thực hiện dự đoán
    \item \textbf{Altair:} Dùng để vẽ các biểu đồ tương tác, đặc biệt là biểu đồ về mức độ ảnh hưởng của đặc trưng lên dự đoán của mô hình
    \item  \textbf{Pandas: }Dùng để xử lý, tạo DataFrame từ dữ liệu đầu vào của người dùng.
\end{itemize}
\subsection{Giao diện và hướng dẫn sử dụng}
\begin{figure}
    \centering
    \includegraphics[width=0.5\linewidth]{image.png}
    \caption{Giao diện chính của ứng dụng web}
    \label{fig:placeholder}
\end{figure}
Ở thanh bên trái, người dùng có thể chọn một trong 4 mô hình đã huấn luyện trước cùng các feature của một chiếc xe, bao gồm:
\begin{itemize}
    \item \textbf{Symboling: }Mức độ rủi ro bảo hiểm của xe. Đây là một thang điểm do các công ty bảo hiểm gán, thường từ -2 (an toàn) đến +3 (rủi ro cao).
    \item \textbf{Fuel type: }Loại nhiên liệu xe sử dụng, gồm gas (xăng) và dầu diesel (Trong dataset không có xe điện).
    \item \textbf{Aspiration: }Công nghệ nạp khí vào động cơ, gồm std (nạp tự nhiên) và turbo (có dùng bộ tăng áp giúp tăng công suất)
    \item \textbf{Door number: }Số lượng cửa xe.
    \item \textbf{Car Body: }Kiểu dáng thân xe (sedan, hatchback, wagon,...)
    \item \textbf{Drive wheel: }Hệ thống dẫn động, ví dụ như: fwd (dẫn động cầu trước), rwd (dẫn đồng cầu sau), 4wd (dẫn động 4 bánh
    \item \textbf{Engine location: }Vị trí đặt động cơ, trước hoặc sau.
    \item \textbf{Engine type: }Kiểu cấu hình động cơ, ví dụ như: ohc (trục cam đơn trên đầu), dohc (trục cam kép).
    \item \textbf{Cylinder number: }Số lượng xi-lanh trong động cơ (4 hoặc 6).
    \item \textbf{Fuel System: }Hệ thống cung cấp nhiên liệu cho động cơ, ví dụ như: mpfi (phun xăng đa điểm), 2bbl (bộ chế hòa khí 2 họng).
    \item \textbf{Car Model: }Tên mẫu xe (audi 100 ls,...), chỉ có thể chọn các mẫu xe có xuất hiện trong dataset.
    \item \textbf{Wheel base: }Khoảng cách giữa trục bánh trước và trục bánh sau của xe.
    \item \textbf{Car length: }Chiều dài tổng thể của xe.
    \item \textbf{Car width: }Chiều rộng tổng thể của xe.
    \item \textbf{Car height: }Chiều cao tổng thể của xe.
    \item \textbf{Curb Weight: }Trọng lượng không tải của xe
    \item \textbf{Engine Size: }Dung tích xi-lanh của động cơ
    \item \textbf{Bore Ratio: }Đường kính của mỗi xi-lanh.
    \item \textbf{Stroke: }Hành trình piston - quãng đường mà piston di chuyển lên xuống bên trong xi-lanh.
    \item \textbf{Compression Ratio: }Tỷ số nén - tỷ lệ giữa thể tích xi-lanh khi piston ở điểm chết dưới và khi ở điểm chết trên.
    \item \textbf{Horsepower: }Công suất cực đại mà động cơ có thể sản sinh theo mã lực.
    \item \textbf{Peak RPM: }Vòng tua máy (số vòng quay mỗi phút - Revolutions Per Minute) mà tại đó động cơ đạt được công suất (horsepower) cực đại.
    \item \textbf{City MPG: }Mức tiêu thụ nhiên liệu trong thành phố - Chỉ số hiệu suất nhiên liệu của xe, được đo bằng số dặm (Miles) xe có thể đi được với một gallon (Gallon) nhiên liệu trong điều kiện lái xe ở đô thị.
    \item \textbf{Highway MPG: }Mức tiêu thụ nhiên liệu trên cao tốc - Chỉ số hiệu suất nhiên liệu của xe (dặm trên mỗi gallon) khi lái xe ở tốc độ ổn định trên đường cao tốc. Chỉ số này thường cao hơn (tiết kiệm nhiên liệu hơn) so với City MPG.
\end{itemize}
Sau khi chọn model và điền các feature của xe, người dùng có thể nhấn nút Predict Price để model thực hiện dự đoán giá. Phần trung tâm của web sẽ là nơi hiển thị kết quá dự đoán của model cùng với một đồ thị chỉ ra mức độ ảnh hưởng của từng feature lên dự đoán của model và một đồ thị chỉ ra các coefficients của các feature trong hàm tuyến tính của mô hình.