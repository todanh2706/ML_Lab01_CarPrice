\pagebreak
\section{Mô tả đồ án và lý do chọn đề tài}
\subsection{Mô tả đồ án}
Đồ án này sẽ tập trung nghiên cứu bài toán xây dựng mô hình hồi quy tuyến tính để dự doán giá xe ô tô dựa trên tập hợp các thuộc tính kỹ thuật và đặc điểm của xe. \\
Sau khi nghiên cứu đặc trưng của tập dữ liệu, nhóm sẽ khám phá 2 hướng tiếp cận chính sau:
\begin{itemize}
    \item Mô hình Chính quy hoá (\textit{Regularizaiton}): Triển khai các kỹ thuật bao gồm Ridge (L2), Lasso (L1) và Elastic Net (kết hợp L1 và L2) nhằm xử lý đa cộng tuyến và lựa chọn thuộc tính thông qua các chuẩn của sai số.
    \item Mô hình Lựa chọn thuộc tính tường minh (\textit{Explicit Feature Selection}): Triển khai mô hình hồi quy tuyến tính cơ bản, nhưng được tối ưu hoá bằng cách áp dụng thuật toán Forward Selection dựa trên Bayesian Information Criterion để chủ động loại bỏ các thuộc tính nhiễu, không cần thiết ngay từ bước tiền xử lý.
\end{itemize}
Đầu tiên, dữ liệu sẽ được chuẩn hóa và chia ra các tập train (60\%), validation (20\%), và test (20\%). Sau đó, các mô hình sẽ được đem train dựa trên tập train hoặc tập train đã được đưa qua BIC trong hướng tiếp cận thứ 2. Với hướng tiếp cận thứ nhất, các thông số sẽ được chọn thông qua grid search. Cuối cùng, các mô hình sẽ được đánh giá dựa trên các chỉ số $R^2$, MAE và RMSE trên tập validation và tập test, nhằm tìm ra mô hình có hiệu quả tốt nhất.
\subsection{Lý do chọn đề tài}
Trong các bài toán hòi quy thực tế, dữ liệu hiếm khi ở trạng thái lý tưởng. Các mô hình OLS (\textit{Ordinary Least Square}) thường hoạt động không tốt khi đối mặt với dữ liệu có hiện tượng đa cộng tuyến (các biến dự đoán tương quan mạnh với nhau). Dẫn đến việc mô hình trở nên thiếu ổn định, có phương sai cao (\textit{high variance}), nhạy cảm với dữ liệu huấn luyện và mất khả năng diễn giải.\\
Do đó, nhóm được thúc đảy bởi mong muốn tìm ra câu trả lời dựa trên dữ liệu thực nghiệm: Liệu việc chấp nhận một lượng độ chệch nhỏ trong các mô hình chính quy hoá có hiệu quả hơn việc loại bỏ hoàn toàn các thuộc tính không? Hay một phương pháp lựa chọn thuộc tính thống kê cẩn thận kết hợp với mô hình đơn giản mới là giải pháp tối ưu? Kết quả của đồ án này sẽ là câu trả lời về sự đánh đổi giữa các kỹ thuật và đưa ra kết luận cụ thể cho bài toán dự đoán trên tập dữ liệu này.
